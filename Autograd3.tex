\documentclass{article}

%title author date
\title{3 自动求导}
\author{LibTorch简单教程}
\date{2025-12-01}

%
\usepackage{ctex}
 
%页边距
 \usepackage[margin=2.5cm]{geometry}
%\geometry{a4paper,left=2cm , right=2cm , top=3cm , bottom=3cm }



%盒子
\usepackage{fancybox}

%代码支持
\usepackage{listings}
\usepackage[usenames,dvipsnames]{xcolor}
\definecolor{mygreen}{rgb}{0,0.6,0}
\definecolor{mygray}{rgb}{0.5,0.5,0.5}
\definecolor{mymauve}{rgb}{0.58,0,0.82}
\lstset{
 backgroundcolor=\color{lightgray},
 basicstyle=\footnotesize,
 breakatwhitespace=false,
 breaklines=true,
 captionpos=b,
 commentstyle=\color{mygreen}\bfseries,
 extendedchars=false,
 frame=shadowbox,
 framerule=0.5pt,
 keepspaces=true,
 keywordstyle=\color{blue}\bfseries,
 language=C++,
 otherkeywords={string},
 numbers=left,
 numbersep=5pt,
 numberstyle=\tiny\color{mygray},
 rulecolor=\color{black},
 showspaces=false,
 showstringspaces=false,
 showtabs=false,
 stepnumber=1,
 stringstyle=\color{mymauve},
 tabsize=2,
 title=\lstname
}

\usepackage{graphics}



\begin{document}

\maketitle

\newpage

\tableofcontents

\newpage

\section{计算图}
\subsection{创建计算图}
假设有一个运算表达式:$f(x,y)=(x+y)(y+1)$,要对$x \texttt{和}	y$分别求偏导数。这是一个多元函数,我们可以把计算过程用图表示出来,这个图就称为计算图。%插入图片 
在代码中,张量经过任何“可微分”的运算后,张量和“运算”都会被记录在这个图中,运算结束后图就创建成功了。有了这个图,我们就可以实现自动求导。自动求导函数会依赖已存在的计算图求解梯度,并且结束后默认销毁这里创建的计算图。



\subsection{节点}
计算过程中,类比多元函数的自变量(叶子节点)和中间变量(非叶子节点),有的张量是被其他张量计算出来的临时变量,这种叫非叶子节点。不由任何张量计算而来,直接存在的叫叶子节点。在计算图中,要重点注意张量的这些属性:是否叶子、是否需要梯度、梯度值、梯度函数。它们可以由这些函数获取:\ovalbox{Tensor::is\_leaf()} 、\ovalbox{Tensor::requires\_grad()} 、 \ovalbox{Tensor::grad()}、 \ovalbox{Tensor::grad\_fn()}。 由叶子节点计算出的节点默认需要梯度。


\subsection{静态图和动态图}
PyTorch是动态图,具有容易理解,灵活的特点。




\section{自动求导函数}
 
自动求导常用的API有2个函数、1个类:\ovalbox{torch::autograd::backward()}、\ovalbox{torch::autograd::grad()}、\ovalbox{torch::autograd::function::Function< >}。
这些是原始的API,我们一般使用\ovalbox{Tensor::backward()}函数完成反向求导 。

\subsection{Tensor::backward()}

函数介绍如下:
原型
\hspace{\fill}
\ovalbox{
\begin{minipage}{0.6\textwidth}
\centering  
inline void 	Tensor::backward( const Tensor \&gradient = \{\}, \\
std::optional<bool> retain\_graph = std::nullopt, \\
bool create\_graph = false, \\
std::optional<TensorList> inputs = std::nullopt   ) const
\end{minipage} 
}
\hspace{\fill}

 功能:自动求解当前标量相对于计算图中所有叶子节点的梯度,否则要指定\ovalbox{gradient}参数。它内部会调用\ovalbox{torch::autograd::backward()}。函数会累计(加法)梯度值到叶子节点,不会自动清零梯度值,所以训练过程需要手动清零梯度值。
 
 参数:一共有四个
\begin{itemize}
\item gradient 非标量时,这个函数会计算雅可比向量积,这个参数接收雅可比矩阵,描述分量的上游梯度关系。当前张量非标量,必须指定某种运算关系,把分量规约为新的标量,新标量对分量的梯度矩阵就是这个参数,也称为上游梯度。

\item retain\_graph 这个函数求导时,依赖已经存在的计算图,求导结束后,会默认销毁该计算图。设置\ovalbox{retain\_graph=true}即可保留,方便重复使用。

\item create\_graph 求出来的梯度是一个张量,它也是被运算出来的,那么就会创建另一张计算图描述梯度和自变量的关系。设置\ovalbox{create\_graph=true}即可保留,方便高阶求导。

\item inputs 我们一般用不到,在API页面有说明。
\end{itemize}

 返回值:void


 \subsection{torch::autograd::grad()}
 
 函数介绍如下:
原型
\hspace{\fill}
\ovalbox{
\begin{minipage}{0.7\textwidth}
\centering  
 variable\_list torch::autograd::grad(const variable\_list \&outputs, \\
 const variable\_list \&inputs, \\
 const variable\_list \&grad\_outputs = \{\}, \\
 std::optional<bool> retain\_graph = std::nullopt, \\
 bool create\_graph = false, \\
 bool allow\_unused = false)
\end{minipage} 
}
\hspace{\fill}

功能:不累计梯度,仅仅计算并返回梯度值。

参数:
\begin{itemize}
\item outputs
\item inputs
\item grad\_outputs
\item retain\_graph 同上
\item create\_graph 同上
\item allow\_unused
\end{itemize}

返回值: \ovalbox{ variable\_list }类型,可以用\ovalbox{torch::Tensor}接收。返回梯度值。

\ovalbox{retain\_graph}和\ovalbox{create\_graph}参数的区别是前者控制函数计算图,后者控制梯度计算图,联系是后者利用前者完成创建,创建后两个图就互相独立,销毁与否都不影响对方。%代码示意参数的关系



还可以局部控制计算图搭建过程的临时变量是否保存。使用\ovalbox{torch::no\_grad()}函数。

\subsection{torch::autograd::function::Function< >}
这是一个模板类。如果我们有额外需求的计算,可以自定义类,公有继承它,重写\ovalbox{forward()}和\ovalbox{backward()}函数,这样自定义运算就变成“可微分运算”了,计算图就能记录此运算,方便自动求导。使用方法在官方仓库中有例子。

\subsection{计算图与自动求导的关系和注意事项}
计算图是由张量计算过程中自动创建的,自动求导函数只是使用现成的计算图,不会创建新的“函数计算图”,而且还会默认销毁这个图。


根据计算图的原理,假设我们已经得到最终变量,下一语句要求导了,在这两语句之间,不可以改变任何图内变量,才能保证结果正确。如果某个变量被改变,计算图会记录改变后的值,导致结果错误。如果想要修改,使用\ovalbox{Tensor::detach()}把变量剥离计算图,得到数据引用,和原变量共享同一块内存,再修改则不会破坏计算图。


\end{document}
