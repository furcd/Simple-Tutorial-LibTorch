\documentclass{article}

%title author date
\title{3 自动求导}
\author{LibTorch简单教程}
\date{2025-12-01}

%
\usepackage{ctex}
 
%页边距
 \usepackage[margin=2.5cm]{geometry}
%\geometry{a4paper,left=2cm , right=2cm , top=3cm , bottom=3cm }



%盒子
\usepackage{fancybox}

%代码支持
\usepackage{listings}
\usepackage[usenames,dvipsnames]{xcolor}
\definecolor{mygreen}{rgb}{0,0.6,0}
\definecolor{mygray}{rgb}{0.5,0.5,0.5}
\definecolor{mymauve}{rgb}{0.58,0,0.82}
\lstset{
 backgroundcolor=\color{lightgray},
 basicstyle=\footnotesize,
 breakatwhitespace=false,
 breaklines=true,
 captionpos=b,
 commentstyle=\color{mygreen}\bfseries,
 extendedchars=false,
 frame=shadowbox,
 framerule=0.5pt,
 keepspaces=true,
 keywordstyle=\color{blue}\bfseries,
 language=C++,
 otherkeywords={string},
 numbers=left,
 numbersep=5pt,
 numberstyle=\tiny\color{mygray},
 rulecolor=\color{black},
 showspaces=false,
 showstringspaces=false,
 showtabs=false,
 stepnumber=1,
 stringstyle=\color{mymauve},
 tabsize=2,
 title=\lstname
}

\usepackage{graphics}



\begin{document}

\maketitle

\newpage

\tableofcontents

\newpage

\section{计算图}
\subsection{创建计算图}
假设有一个运算表达式:$f(x,y)=(x+y)(y+1)$,要对$x \texttt{和}	y$分别求偏导数。这是一个多元函数,我们可以把计算过程用图表示出来,这个图就称为计算图。%插入图片 
在代码中,张量经过任何“可微分”的运算后,张量和“运算”都会被记录在这个图中,运算结束后图就创建成功了。有了这个图,我们就可以实现自动求导。自动求导函数会依赖已存在的计算图求解梯度,并且结束后默认销毁这里创建的计算图。



\subsection{节点}
计算过程中,类比多元函数的自变量(叶子节点)和中间变量(非叶子节点),有的张量是被其他张量计算出来的临时变量,这种叫非叶子节点。不由任何张量计算而来,直接存在的叫叶子节点。在计算图中,要重点注意张量的这些属性:是否叶子、是否需要梯度、梯度值、梯度函数。它们可以由这些函数获取:\ovalbox{Tensor::is\_leaf()} 、\ovalbox{Tensor::requires\_grad()} 、 \ovalbox{Tensor::grad()}、 \ovalbox{Tensor::grad\_fn()}。 由叶子节点计算出的节点默认需要梯度。


\subsection{静态图和动态图}
PyTorch是动态图,具有容易理解,灵活的特点。




\section{自动求导函数}
自动求导
自动求导常用的API有2个函数、1个类:\ovalbox{torch::autograd::backward()}、\ovalbox{torch::autograd::grad()}、\ovalbox{torch::autograd::function::Function}。

\ovalbox{backward()}函数介绍如下:

功能:自动求解计算图中所有节点的梯度。我们一般使用\ovalbox{Tensor::backward()}函数求导,它内部会调用\ovalbox{torch::autograd::backward()}。

参数:

返回值:

\ovalbox{backward()}函数会累计梯度值到叶子节点,不会自动清零梯度值,所以训练过程需要手动清零梯度值。
\ovalbox{grad()}函数不累计梯度,仅仅计算并返回梯度值。

\ovalbox{retain\_graph}和\ovalbox{create\_graph}参数的区别是前者控制函数计算图,后者控制梯度计算图,联系是后者利用前者完成创建,创建后两个图就互相独立,销毁与否都不影响对方。%代码示意参数的关系

根据计算图的原理,在求导语句和最终变量语句之间,不可以改变任何变量,才能保证结果正确。如果某个变量被改变,计算图会记录改变后的值,导致结果错误。如果想要修改,使用\ovalbox{Tensor::detach()}把变量剥离计算图,得到数据引用,和原变量共享同一块内存,但是修改不会破坏计算图。

还可以局部控制计算图搭建过程的临时变量是否保存。使用\ovalbox{torch::no\_grad()}函数。


\end{document}
