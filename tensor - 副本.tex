\documentclass[a3paper]{article}

%中文支持
\usepackage{ctex}


%盒子
\usepackage{fancybox}

%代码支持
\usepackage{listings}
\usepackage[usenames,dvipsnames]{xcolor}
\definecolor{mygreen}{rgb}{0,0.6,0}
\definecolor{mygray}{rgb}{0.5,0.5,0.5}
\definecolor{mymauve}{rgb}{0.58,0,0.82}
\lstset{
 backgroundcolor=\color{lightgray},
 basicstyle=\footnotesize,
 breakatwhitespace=false,
 breaklines=true,
 captionpos=b,
 commentstyle=\color{mygreen}\bfseries,
 extendedchars=false,
 frame=shadowbox,
 framerule=0.5pt,
 keepspaces=true,
 keywordstyle=\color{blue}\bfseries,
 language=C++,
 otherkeywords={string},
 numbers=left,
 numbersep=5pt,
 numberstyle=\tiny\color{mygray},
 rulecolor=\color{black},
 showspaces=false,
 showstringspaces=false,
 showtabs=false,
 stepnumber=1,
 stringstyle=\color{mymauve},
 tabsize=2,
 title=\lstname
}



%文档信息 标题作者日期
\title{2 张量}
\author{LibTorch简单教程}
\date{2025-11-21}

\begin{document}

	\maketitle
	
	\newpage
	
	\section{张量模型}
		张量在LibTorch中表示为一种模型(scheme),有三个常用属性,分别是 $Value \quad Size \quad TensorOptions$ ,这里介绍的三个属性名称并非一一对应源代码,仅作理解使用。
		但是$TensorOptions$是源码真实存在的,可以在官方文档找到详细说明。
		\subsection{Value}
		$Value$ 属性是指张量中存储的数值。众所周知,计算机把数字、文本、图像等人类理解的事物一律表示为二值编码(电压高低离散值)。而人类又使用“二进制计数系统”理解这些二值,所以它们就成为二进制数字了。更底层的一步理解是,使用“布尔逻辑代数”理解这些编码。所以由此可以得出,$Value$ 数值包括普遍意义的“数值数字”,也包含字符串文本、图像等。
				 
		$Value$ 的深层含义解释完了,我们接下来看一下实际使用中的场景。第一个是我们可以从C++原生数值类型转换得到,比如{\ovalbox{int} \ovalbox{float} \ovalbox{double}}等。
		
		第二个是可以从STL数据转化,比如 \ovalbox{std::vector::data} 属性获取数据指针。
		
		第三个就从第三方库转化,比如 \ovalbox{cv::Mat::data}属性获取矩阵或者图像数据指针,从OpenCV转化要注意数值归一化,通道顺序符合神经网络模型输入顺序。
		
		\subsection{Size}
		$Size$ 属性是指张量的维度数量和每个维度索引范围。比如,$Size = 4 *5* 8$ 意味着这个张量{\ovalbox{维度=3}因为有三个数字(或者说两个分割标记),每个维度分配编号或者索引分别为\ovalbox{0、1、2};维度索引范围:\ovalbox{第0号维度} 取值\ovalbox{0-1-2-3},\ovalbox{第1号维度} 取值\ovalbox{0-1-2-3-4},\ovalbox{第2号维度} 取值\ovalbox{0-1-2-3-4-5-6-7}。这是编程时的用法。通俗的理解就是4行5列8通道图像。
		
		
		\subsection{TensorOptions}
		$TensorOptions$ 属性是一个\ovalbox{class},定义了多个子属性,具体可以查看源码和文档。这里介绍常用的三个子属性:\ovalbox{$dtype$} \ovalbox{$grad$} \ovalbox{$device$}。
		
		\ovalbox{$dtype$}是指 $Value$ 的存储方式。如果从\ovalbox{int}转化而来,则 \ovalbox{$dtype=kInt$},同理\ovalbox{$float \rightarrow kFloat$}\ovalbox{$double \rightarrow kDouble$}。\ovalbox{$dtype$}属性可以有其它取值,参看文档和源码。
		\ovalbox{$grad$} 是指在自动微分时该 \ovalbox{$Tensor$} 是否保留梯度。	  
		\ovalbox{$device$}是指 \ovalbox{$Tensor$} 在哪个设备运算。常见取值有\ovalbox{$device=kCPU$} \ovalbox{$device=kCUDA$}。
	
\section{张量创建}
		\subsection{工厂函数}
		LibTorch提供了一系列函数快速创建特定的张量,称之为“工厂函数”。详情参看文档。
		
		\subsection{手动指定}
		我们使用\ovalbox{torch::tensor(arg1, arg2)}函数创建,需要同时手动设置$Value$ 、$  Size $ 、 $ TensorOptions$属性。\ovalbox{arg1}同时接收$Value$和$Size$,\ovalbox{arg2}接收$TensorOptions$。其中重点是前两个属性$Value$、$ Size$,$Value$非常容易理解,就是诸如$1.5 \quad 2 \quad 3.0$这样不同的值。
		
		详细介绍$Size$设置。这个属性是实参\ovalbox{std::initializer}推算而来。我们注意观察配对的\ovalbox{\{ $\quad$ \}} 和 \ovalbox{,}。
		\begin{itemize}
		\item 3 $\rightarrow$ $Size=null$ 
		\item \{$\quad$\} $\rightarrow$ $Size=0$ 
		\item \{789\} $\rightarrow$ $Size=1$ 
		\item \{423, 9213, 2647\} $\rightarrow$ $Size=3$ 
		\item \{\{1,2,3\},\{7,8,9\}\} $\rightarrow$ $Size=2*3$
		\item \{\{1,2,3,4,5\},\{7,8,9,10,11\}\} $\rightarrow$ $Size=2*5$
		\item \{\{1,2,3,4,5\},\{7,8,9,10,11\},\{21,22,23,24,25\}\} $\rightarrow$ $Size=3*5$
		\item \{\\\{\{1\},\{2\},\{3\},\{4\},\{5\}\},\\     \{\{7\},\{8\},\{9\},\{10\},\{11\}\},\\     \{\{21\},\{22\},\{23\},\{24\},\{25\}\}\\\} $\rightarrow$ $Size=3*5*1$
		\item \{ \\
		 		\{\{000,001\},$\quad$\{010,011\},$\quad$\{020,021\},$\quad$\{030,031\},$\quad$\{040,041\}\},
		\\
				\{\{100,101\},$\quad$\{110,111\},$\quad$\{120,121\},$\quad$\{130,131\},$\quad$\{140,141\}\},
		\\
				\{\{200,201\},$\quad$\{210,211\},$\quad$\{220,221\},$\quad$\{230,231\},$\quad$\{240,241\}\}
				\\\} $\rightarrow$ $Size=3*5*2$				
		\end{itemize}
		
		
		我们可以发现配对的\ovalbox{\{ $\quad$ \}}——既有\ovalbox{\{  }又有\ovalbox{  \}}才算配对——会创建一个维度,同时\ovalbox{,}则分割每个维度自己负责的元素。
		例如$Size=2*3$ \ovalbox{0号维度}的元素有2个分别是 \ovalbox{\{1,2,3\}}和\ovalbox{\{7,8,9\}},这两个元素被1个数量的\ovalbox{,}隔开。
		它的单个元素由配对花括号\ovalbox{\{ $\quad$ \}}和\ovalbox{,}互相协作产生,类比1个大集合包含3个小集合,那么大集合的元素就成为了小集合,每个小集合包含5个实数,小集合的元素就成为实数而不是集合。
		可见元素是一个相对而言的概念。
		至此,我们可以总结出规律。\ovalbox{\{ $\quad$ \}}创建维度,而 \ovalbox{,}确定“元素”数量并且分隔它们。
		
		$Size=5*2*4$ 的张量应该怎么写呢?
		
		这个张量又是什么形状?\\
     \ovalbox{
		          \{  \{\{   ,    ,   \}, \{   ,    ,   \} ,  \{   ,    ,   \}  ,    \{   ,    ,   \}  ,    \{   ,    ,   \}\}, 
		          $\quad$	    
		              \{\{   ,    ,   \}, \{   ,    ,   \} ,  \{   ,    ,   \}  ,    \{   ,    ,   \}  ,    \{   ,    ,   \}\}   \}
		        }
		
		
		\subsection{数据指针转换}
		到这里就比较简单了,使用函数\\ \ovalbox{torch::from\_blob(viod* $Value$ , at::IntArrayRef $Size$ , const at::TensorOptions\& )}
		
		在网上可以找到使用方式,
		其中实参传递给$Size$时,是\ovalbox{std::initializer}格式——注意,这个和上节的推算不一样,只是长得像——比如$Size=3*5*2$ $\rightarrow$ \ovalbox{\{ 3, 5, 2 \}},
		这就确认了张量的$Size$。当只有一个维度时,可以省略花括号,$Size=3$ $\rightarrow$ \ovalbox{\{ 3 \}} $\iff$ \ovalbox{  3  }。
		
		总之,传参写法就是,花括号包起来并且分隔改为逗号。可以发现,这种方式创建维度$null$的张量传入\ovalbox{\{   \}}即可。其实在官方文档可以查到,$null$其实就是标量。
		
		思考一下,我们把视角脱离,不再执着符号外貌。
		当我们看见类似  \{\{\{   ,    ,   \}, \{   ,    ,   \}\},\{\{   ,    ,   \}, \{   ,    ,   \}\}\}符号时,以张量推算视角和$Size $传参视角会有什么不同的结果?
		参看下面代码,欢迎补充内容。
			 
		 \begin{lstlisting}[caption =\{ C++代码示例 \}  ]
# include <iostream>

# include <torch/torch.h>


int main()
{
	float a[800] = { 2,  3,   .1 };

	//无括号
	std::cout << torch::from_blob(a, 0) << std::endl;
	std::cout << torch::from_blob(a, 1) << std::endl;
	std::cout << torch::from_blob(a, 2) << std::endl;
	//有括号
	std::cout << torch::from_blob(a, {  }) << std::endl;
	std::cout << torch::from_blob(a, { 2 }) << std::endl;
	std::cout << torch::from_blob(a, { 2,5 }) << std::endl;
	std::cout << torch::from_blob(a, { 2,5,7 }) << std::endl;
	//嵌套括号
	std::cout << torch::from_blob(a, {  }) << std::endl;
	std::cout << torch::from_blob(a, { {} }) << std::endl;
	std::cout << torch::from_blob(a, { {{}} }) << std::endl;
	/*
	std::cout << torch::from_blob(a, {{{{}}}}) << std::endl; 
	会报错,最多嵌套三层
	*/
	std::cout << ( a[1] = { } ) << std::endl;
	std::cout << torch::from_blob(a, { 0  , }) << std::endl;  //末尾逗号
	std::cout << torch::from_blob(a, { { } , }) << std::endl; //末尾逗号
	std::cout << torch::from_blob(a, { { } , {} }) << std::endl;
	std::cout << torch::from_blob(a, { {    } , {} , {}   }) << std::endl;
	/* 
	std::cout << torch::from_blob(a, { { {} } , {} , {} , }) << std::endl;  
	报错,运算符优先级问题
	*/

	return 0;
}
		 \end{lstlisting}
		
		\subsection{转换为非张量}
		使用\ovalbox{Tensor::data\_ptr<type>()}获取张量数据指针,\ovalbox{type}想要的数据类型。还有一个函数\ovalbox{Tensor::item<type>()}只针对标量使用。
		
		\subsection{使用构造函数}
		构造函数\ovalbox{torch::Tensor()}可以直接初始化张量,它有多个重载,参见官方文档。这里涉及一个深浅拷贝问题。
		赋值构造是浅拷贝,内存中的数据只有单份,一荣俱荣一损俱损。使用\ovalbox{Tensor::clone()}函数创建全新的内存空间,以单独存储数据。
		
		

	\section{张量计算}

	
	\section{张量索引}


	\section{张量形状操作}


	\section{张量互相操作}


	\section{张量小结}












\end{document}