\documentclass{article}

%标题作者日期
\title{1 绪论}
\author{LibTorch简单教程}
\date{2025-11-23}

%中文支持
\usepackage{ctex}

%other packages
\usepackage{fancybox}

\usepackage{hyperref}
\hypersetup
{
   hidelinks,
   colorlinks=true,
   allcolors=blue,
   pdfstartview=Fit,
   breaklinks=true
}

\usepackage{graphicx}
\usepackage{svg}






\begin{document}	

\maketitle

\newpage

\section{前言}
本教程简单学习LibTorch用法。
所谓学习一个库,可以设计如下评价标准:
\begin{itemize}
\item  1)会搜索库的相关资料;
\item 2) 了解库的设计思路
\item  3)  掌握库函数用法;
\item 4) 调试解决报错
\end{itemize}

\section{查资料}
\subsection{资料类型}
俗话说三生万物。学会这个库要收集必需材料:
\begin{itemize}
\item  1)原料:官方文档、库源码、附带例程;
\item  2)解惑:提问AI、大家的各种人脉;
\item  3)分享:第三方资料如书籍、论坛、交流群等;
\item  4)重要:心得总结之资料。
\end{itemize}

\subsection{原始资料}
原始资料就是库作者提供的第一手信息。
一般是官方文档、源代码、示例程序三个部分。 
 
 
官方文档:

$\quad$\href{  https://docs.pytorch.org/docs/stable/cpp_index.html    }{C++接口介绍页面}

$\quad$\href{  https://docs.pytorch.org/cppdocs/    }{API页面}

$\quad$\href{   https://docs.pytorch.ac.cn/tutorials/advanced/cpp_frontend.html }{官方教程}

 
库源码、例程:

$\quad$\href{https://github.com/pytorch/examples/tree/main/cpp   }{源代码和示例程序}


\subsection{解惑资料}
AI提问、认识的人也在搞这个,动用你的人脉都可以。

\subsection{分享资料}
第三方资料,如书籍、论坛、交流群等

$\quad$\url{  https://github.com/AllentDan/LibtorchTutorials }

$\quad$\url{ https://zhuanlan.zhihu.com/p/609288586  }

$\quad$\url{  https://dataxujing.github.io/libtorch_tutorials }

\includesvg[width=0.40\textwidth]{C:/Users/vic/Pictures/Screenshots/f1}\includesvg[width=0.40\textwidth]{C:/Users/vic/Pictures/Screenshots/f2.svg}

\subsection{重要资料}
心得总结之资料。自己学习、编程中的经验总结 .

\section{库结构}
库结构参考API页面顶部第二行的main项

本API大致分为五个部分:
\begin{itemize}
\item ATen: 张量支持、张量数学运算、所有的张量及运算最终都归结到这里。

\item Autograd: 给ATen提供自动微分功能。

\item C++ Frontend: 训练和评估模型接口、性能高。

\item TorchScript: JIT接口、加载模型就用这个。

\item C++ Extensions: 高级模块、自定义接口、自定义CUDA行为

\end{itemize} 




\section{库函数}
\subsection{库函数参考}
接口原型在API页面顶部第二行的Library API项。
用法在各个部分都有。
重要:LibTorch和PyTorch接口相似,可以参考PyTorch资料的用法,在Library API查找。


\subsection{库函数学习标准}
\begin{itemize}
\item 1)常用的函数接口要熟悉功能、返回值、参数、注意事项
\item 2)使用过程出现BUG,会调试、分析、解决,并总结规范用法
\item 3)论文中遇见张量操作、算法,会编写对应张量代码


\end{itemize} 

\subsection{库函数BUG调试}

\begin{itemize}
\item 1)检查环境配置
\item 2)代码错误:编译错误、运行错误、逻辑错误、拼写错误
\item 3)使用try catch
\item 4)打印输出,定位错误位置 
\item 5)Debug模式

\end{itemize}  
 

\section{环境配置}
\subsection{基本环境}
Windows 11    C++   VS2022   LibTorch CPU

Windows 11    C++   VS2022   LibTorch GPU

\subsection{额外库}
OpenCV库,是本教程使用的,同时也能百度到结合LibTorch配置环境的教程




\section{小结}
1)工欲善其事必先利其器,资料齐全,IDE熟练



2)动态更新经验,多交流


3)这个库参考资料很少,自己要多总结分享





\end{document}	